%%%%%%%%%%%%%%%%%%%%%%%%%%%%%%%%%%%%%%%%%
% Journal Article
% LaTeX Template
% Version 1.3 (9/9/13)
%
% This template has been downloaded from:
% http://www.LaTeXTemplates.com
%
% Original author:
% Frits Wenneker (http://www.howtotex.com)
%
% License:
% CC BY-NC-SA 3.0 (http://creativecommons.org/licenses/by-nc-sa/3.0/)
%
%%%%%%%%%%%%%%%%%%%%%%%%%%%%%%%%%%%%%%%%%

%----------------------------------------------------------------------------------------
%	PACKAGES AND OTHER DOCUMENT CONFIGURATIONS
%----------------------------------------------------------------------------------------

\documentclass[twoside]{article}

\usepackage{lipsum} % Package to generate dummy text throughout this template

\usepackage[sc]{mathpazo} % Use the Palatino font
\usepackage[utf8]{inputenc}
\usepackage[T1]{fontenc} % Use 8-bit encoding that has 256 glyphs
\linespread{1.05} % Line spacing - Palatino needs more space between lines
\usepackage{microtype} % Slightly tweak font spacing for aesthetics

\usepackage[hmarginratio=1:1,top=32mm,columnsep=20pt]{geometry} % Document margins
\usepackage{multicol} % Used for the two-column layout of the document
\usepackage[hang, small,labelfont=bf,up,textfont=it,up]{caption} % Custom captions under/above floats in tables or figures
\usepackage{booktabs} % Horizontal rules in tables
\usepackage{float} % Required for tables and figures in the multi-column environment - they need to be placed in specific locations with the [H] (e.g. \begin{table}[H])
\usepackage[hidelinks]{hyperref} % For hyperlinks in the PDF

\usepackage{lettrine} % The lettrine is the first enlarged letter at the beginning of the text
\usepackage{paralist} % Used for the compactitem environment which makes bullet points with less space between them

\usepackage{amsmath,amssymb,amsfonts}
\providecommand{\abs}[1]{\left \lvert #1 \right \rvert}

\usepackage{abstract} % Allows abstract customization
\renewcommand{\abstractnamefont}{\normalfont\bfseries} % Set the "Abstract" text to bold
\renewcommand{\abstracttextfont}{\normalfont\small\itshape} % Set the abstract itself to small italic text

\usepackage{titlesec} % Allows customization of titles
\renewcommand\thesection{\Roman{section}} % Roman numerals for the sections
\renewcommand\thesubsection{\Roman{subsection}} % Roman numerals for subsections
\titleformat{\section}[block]{\large\scshape\centering}{\thesection.}{1em}{} % Change the look of the section titles
\titleformat{\subsection}[block]{\large}{\thesubsection.}{1em}{} % Change the look of the section titles

\usepackage{fancyhdr} % Headers and footers
\pagestyle{fancy} % All pages have headers and footers
\fancyhead{} % Blank out the default header
\fancyfoot{} % Blank out the default footer
%\fancyhead[C]{Running title $\bullet$ November 2012 $\bullet$ Vol. XXI, No. 1} % Custom header text
\fancyfoot[RO,LE]{\thepage} % Custom footer text

%----------------------------------------------------------------------------------------
%	TITLE SECTION
%----------------------------------------------------------------------------------------

\title{\vspace{-15mm}\fontsize{24pt}{10pt}\selectfont\textbf{Local Adaptive
Optimization of Time step}} % Article title

\author{
\large
\textsc{Malte Stær Nissen}\thanks{Thank to my supervisor Sune Darkner for
helping me realize this project}\\[2mm] % Your name
\normalsize University of Copenhagen \\ % Your institution
\normalsize \href{mailto:nissen@diku.dk}{nissen@diku.dk} % Your email address
\vspace{-5mm}
}
\date{}

%----------------------------------------------------------------------------------------

\begin{document}

\maketitle % Insert title

\thispagestyle{fancy} % All pages have headers and footers

%----------------------------------------------------------------------------------------
%	ABSTRACT
%----------------------------------------------------------------------------------------

\begin{abstract}


\end{abstract}

%----------------------------------------------------------------------------------------
%	ARTICLE CONTENTS
%----------------------------------------------------------------------------------------

\begin{multicols}{2} % Two-column layout throughout the main article text

\section{Introduction}
\lettrine[nindent=0em,lines=3]{S} imulations of large hyper elastic materials
are very time consuming. When working with materials of varying density
of vertices/scale of mesh, the highest densities often make the most
strict bounds on size of the time step of the simulation in order to keep
the materials (and grids) stable and the discretization correct. in a
``standard'' simulation with a global time step size we need to compute
possibly unnecessarily many computations on the coarser parts of the material
caused by this bound. When furthermore having materials with a large amount
of somewhat equally distributed vertices and smaller patches of detailed
areas with higher density of vertices, it would be an advantage to be able
to perform large time steps for the majority of the material and smaller
steps for the local patches where vertices could cause instability. we call
the concept of varying the timestep local adaptive optimization of time step
(LAOT), in which local either refer to spatial locality or time locality. We
will in this report study LAOT in the simple 1 dimensional simulation of a
mass spring system in order to gain knowledge into the field for further use
in simulation of hyper elastic materials. We will first describe the simulation
problem and then go into the study of the application of LOAT herein.

%------------------------------------------------

\section{Previous work}
As according to \cite{Gander:2013} local time stepping was first studied
in the community of ordinary differential equations (ODE) with Rice
\cite{rice:1960} developing the split Runge-Kutta methods (multi rate
Runge-Kutta methods). The concept behind these methods is the splitting of
the ODEs into multiple (two components in \cite{rice:1960}) components of
which each component needs to be integrated in different scales. A strategy
is chosen as proposed in \cite{Kvaernoe:1999} of either computing the coarser
part first (``Fastest first strategy'') followed by the finer part or vice
versa (``Slowest first strategy''). In order to perform these sequential
computations either interpolation or extrapolation of the first computation
is performed to compute the second part in each time step and the different
time step sizes can be adapted to fit the model in each step as well. See
\cite{Kvaernoe:1999} and \cite{Gear:1984} (similar strategy for linear multi
step methods) for more detailed descriptions of the two strategies.

In the partial differential equations (PDE) community, the adaptive time
stepping area was explored and developed with the goal of simulating
very specific known problems such as the (hyperbolic) wave equations
and (parabolic) heat equations instead of more general applications. We
will adhere from describing the methods further in this report, but a
short description and comparison of the specific methods can be found in
\cite{Gander:2013}. Generally the methods developed at first all make use
of the same basic concepts for making the local adaptive time stepping:
Interpolation, extrapolation, prediction and correction, which we are going to
use in the work of this report as well.

Recently newer and faster methods with local time stepping have been developed
on the basis of nonlinear PDEs and the work performed in this field in
the 80s and 90s. First the multi resolution (MR) schemes for creating
space-adaptive discretizations and refinements of these were developed, see
\cite{Berger:1984}. Then multiple MR based schemes were developed. Domingues
et al. \cite{Domingues:2008} is one of the more recent of these MR schemes.
They describe a local scale-dependent time stepping for a space-adaptive multi
resolution scheme using the finite volume method in order to obtain speed-up
using larger time-steps without violating a defined stability constraints,
which is essentially the same motivation as this report. The method is based
on an explicit Runge-Kutta method of second order. As expected the time step
size is imposed by a stability condition of the explicit Runge-Kutta on the
finest scale, which increases with the scale of the mesh and hence we are able
to increase the time step as well without violating the stability condition.


%------------------------------------------------

\section{Methods}

We will in this report consider the 1 dimensional problem of simulating a mass
spring system, which is usually (in 3 dimensions) used for simulating
cloth and hair etc. A mass spring system is a system consisting of a set of
vertices $V$ each having a specified mass $m$, force $F$ and velocity $u$ and
a set of springs $S$ with two endpoints $p_f \in V$ and $p_t \in V \setminus
p_f$. Each spring $s$ has a damping factor $d_s$, a stiffness constant $K_s$ and a rest
length $l_{s0}$ describing the spring's physical properties.

Things to write about (notes):

Damping force $F_d$:
\begin{align}
    \Delta l &= l_{left} - l_{right} \\
    F_d      &= - C_d v(i) =  - C_d \frac{\left(v_{left} - v_{right}\right)
\Delta l}{\abs{\Delta l}}
\end{align}

Spring force $F_s$:
\begin{align}
    F_s = -kx = -k \left( l - l_0 \right)
\end{align}

Explicit Euler method:
\begin{align}
    x(t+h) &= x(t) + h \frac{\partial x(t)}{\partial t}
\end{align}

Midpoint method:
\begin{align}
    x(t+h) = x(t) + h \left( f \left(x + \frac{h}{2} f \left(t \right) \right) \right)
\end{align}

\cite{Keshav:2007}
\subsection{Switch}
Change timestep when detecting collision/switch in next iteration

\subsection{Spatial LAOT}
Atomic update

Fixed update rate and detection of need for re-calculation of individual springs

Heuristic: Length of spring, steps since last update

\subsection{Inverse}
Time-local optimization according to heuristic:

\begin{align}
    \frac{1}{a} &= \frac{1}{\frac{-kx}{m}} = \frac{m}{-kx} \\
    \Delta t &= \abs{\frac{1}{a}} + \frac{1}{0.1}
\end{align}

%------------------------------------------------

\section{Results}

%\begin{table}[H]
%\caption{Example table}
%\centering
%\begin{tabular}{llr}
%\toprule
%\multicolumn{2}{c}{Name} \\
%\cmidrule(r){1-2}
%First name & Last Name & Grade \\
%\midrule
%John & Doe & $7.5$ \\
%Richard & Miles & $2$ \\
%\bottomrule
%\end{tabular}
%\end{table}

%------------------------------------------------

\section{Discussion}

Adaptive timestepping based on highest acceleration/smallest grid

%----------------------------------------------------------------------------------------
%	REFERENCE LIST
%----------------------------------------------------------------------------------------

\bibliography{bibliography}
\bibliographystyle{plain}

%----------------------------------------------------------------------------------------

\end{multicols}

\end{document}
