\documentclass[11pt,a4paper]{article}
\usepackage[english]{babel}
\usepackage[utf8]{inputenc}
\usepackage[T1]{fontenc}
\usepackage{lmodern}

\usepackage{amsmath,amssymb,amsfonts}

\title{Project Contract\\Local Adaptive Optimization of Timestep}
\author{Malte Stær Nissen\\University of Copenhagen\\\texttt{malte.nissen@gmail.com}\\Supervisor: Sune Darkner\\\texttt{darkner@diku.dk}}

\begin{document}
\maketitle

\section{Problem Statement}
Study the application of locally adaptive timestepping in 1 dimensional
simulation of a mass spring system.

\section{Learning Goals}
Course specific learning goals:
\begin{enumerate}
	\item Formulate an operational project plan.
	\item Search the relevant literature and write a literature review setting own work in perspective.
	\item Be able to formulate what plagiarism is, and demonstrate proper citation and reference style.
	\item Solve a selected problem of fair Computational and Mathematical Modelling / eScience content and difficulty.
	\item Produce a thorough experiment plan that clearly demonstrates the quality of and highlights boundaries for the solution. 
	\item Produce a scientific text of fair quality both textually and scientifically.
	\item The student must be able to make an oral presentation of own work.
\end{enumerate}
Project specific learning goals:
\begin{enumerate}
    \item Implement local adaptive optimization of timesteps in a 1
dimensional mass-spring system
    \item Develop and implement possible heuristics for use in LAOT schemes
	\item Evaluate LAOT to minimize the risk of unwanted deformations in grids
        and minimize the computational cost.
    \item Discuss the use and combination of LAOT schemes and heuristics based
        on the evaluation of these
\end{enumerate}

\section{Problem Description}

\subsection{Motivation}
When running a simulation of large hyperelastic materials we wish to make
large enough steps for the simulation to be performed within a reasonable
timeframe, but we still want the simulation to be accurate and to avoid
the mesh from curling up or exploding due to too large determinants of the
moving mesh vertices. In some hyperelastic materials most of the
discretized points are able to be simulated using larger timesteps whereas
smaller areas might need smaller timesteps. These situations are the ones
we wish to accomodate in this paper by gaining insight into the area of LAOT in
a simple 1 dimensional simulation (mass spring system).

\subsection{Proposed solution}
We wish to develop a method for locally adapting the timestep size (either
spatial or in time) in
order to avoid curl ups/explosions for minimizing the number of computations
needed for a simulation. We do this by exploring possible methods and heuristics
for LAOT application without affecting the outcome of a standard simulation.

\subsection{Success criteria}
To gain insight into the application of LAOT.

\subsection{Risk assessment}
Medium. The technique of LAOT has been performed before in various shapes but
not the same as our proposed solution.

\end{document}
