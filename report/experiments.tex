\documentclass[11pt,a4paper]{article}

\usepackage[utf8]{inputenc}
\usepackage[english]{babel}
\usepackage[T1]{fontenc}
\usepackage{lmodern}

\usepackage{amsmath,amssymb,amsfonts}

\title{Experimental plan}
\author{Malte Stær Nissen}

\begin{document}
\maketitle

The experimental plan is overall split in two parts: 1 dimensional experiments and 2
dimensional experiments.

\subsection{1 dimensional experiments}
We are working with the problem of simulating a spring mass particle system in 1
dimension. Our hypothesis is that we are able to apply adaptive timestepping to
the simulation and thereby decrease the number of computations needed to
simulate the system.

We first experiment with a uniform 1D mesh where we manually choose different
timestep sizes to test for possible side-effects in as simple a simulation as
possible.

We then perform the adaptive timestepping on a 1D mesh having coarse and finer
parts.

\subsection{2 dimensional experiments}

In 2 dimensions we are working with the problem of simulating hyperelastic
materials using adaptive timestepping as for the 1 dimensional experiments.

We split the 2D experiments in a test on a somewhat uniform 2D mesh having
manually chosen some triangles to be simumlated with smaller timesteps than
others. After these experiments we move on to performing 2D experiments on
space-adaptive meshes having both coarse and fine parts.


\subsection{Comparative experiments}

Finally we are going to compare the performance of the adaptive timestepping
with regular timestepping schemes in 2 dimensions.



%Lav 1 dimensionelle problemer først med fjedrer
%
%Til triangulering:
%qmesh,
%delaunay,
%
%Store trekanter med små trekanter i midten.
%
%Fast forhold mellem tidsskridt til at starte med.
%
%Derefter detekt når advektionen går galt.
%Når arealet er lig 0
%Krydsprodukt til udregningerne.
%
%$\frac{\alpha t}{2} < \Delta t$
%
%linære beregninger
%
%Samtidig timestepping uden genberegning på store skridt
%
%Forsøg at se hvor meget vi kan presse citronen.
%
%Benyt eget lille framework og modificer det
%
%
%Evt. benyt lokale timesteps med genberegninger løbende



\end{document}

