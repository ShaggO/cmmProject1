\documentclass[11pt,a4paper]{article}

\usepackage[utf8]{inputenc}
\usepackage[english]{babel}
\usepackage[T1]{fontenc}
\usepackage{lmodern}

\usepackage{amsmath,amssymb,amsfonts}

\title{}
\author{Malte Stær Nissen}

\begin{document}
\maketitle

Til triangulering:
qmesh,
delaunay,

Store trekanter med små trekanter i midten.

Fast forhold mellem tidsskridt til at starte med.

Derefter detekt når advektionen går galt.

$\frac{\alpha t}{2} < \Delta t$

linære beregninger

Samtidig timestepping uden genberegning på store skridt

Forsøg at se hvor meget vi kan presse citronen.

Benyt eget lille framework og modificer det


Evt. benyt lokale timesteps med genberegninger løbende

\end{document}

