\documentclass[11pt,a4paper]{article}
\usepackage[english]{babel}
\usepackage[utf8]{inputenc}
\usepackage[T1]{fontenc}
\usepackage{lmodern}

\usepackage{amsmath,amssymb,amsfonts}

\title{Project Contract\\Local Adaptive Optimization of Timestep}
\author{Malte Stær Nissen\\University of Copenhagen\\\texttt{malte.nissen@gmail.com}\\Supervisor: Sune Darkner\\\texttt{darkner@diku.dk}}

\begin{document}
\maketitle

\section{Learning Goals}
Course specific learning goals:
\begin{enumerate}
	\item Formulate an operational project plan.
	\item Search the relevant literature and write a literature review setting own work in perspective.
	\item Be able to formulate what plagiarism is, and demonstrate proper citation and reference style.
	\item Solve a selected problem of fair Computational and Mathematical Modelling / eScience content and difficulty.
	\item Produce a thorough experiment plan that clearly demonstrates the quality of and highlights boundaries for the solution. 
	\item Produce a scientific text of fair quality both textually and scientifically.
	\item The student must be able to make an oral presentation of own work.
\end{enumerate}
Project specific learning goals:
\begin{enumerate}
	\item Evaluate local adaptive optimization of timesteps to minimize the risk of unwanted deformations in grids.
\end{enumerate}

\section{Problem Description}

\subsection{Problem Statement}

\subsection{Motivation}
When running a simulation of large hyperelastic materials we wish to make
large enough steps for the simulation to be performed within a reasonable
timeframe, but we still want the simulation to be accurate and to avoid
the mesh from curling up or exploding due to too large determinants of the
moving mesh vertices. In some hyperelastic materials most of the
discretized points are able to be simulated using larger timesteps whereas
smaller areas might need smaller timesteps. These situations are the ones
we wish to accomodate in this paper.

\subsection{Proposed solution}
We wish to develop a method for locally adapting the timestep size in
order to avoid curl ups/explosions while still maintaining the original
timestep size of the rest of the simulation. In other words for each of
the original timesteps we perform further simulations of problematic local
areas of the grid with smaller timesteps.

\subsection{Success criteria}
Avoid curl ups/explosions of the grid while maintaining the original
stepsize by using local subsimulations (with smaller stepsizes)

\subsection{Risk assessment}
Low. The technique has been performed before in various shapes and it
should therefore be do-able.

\end{document}
