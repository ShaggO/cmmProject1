\documentclass[11pt,a4paper]{article}
\usepackage[english]{babel}
\usepackage[utf8]{inputenc}
\usepackage[T1]{fontenc}
\usepackage{lmodern}
\usepackage{todonotes}

\usepackage{amsmath,amssymb,amsfonts}

\providecommand{\abs}[1]{\left \lvert #1 \right \rvert}

\title{Local Adaptive Optimization of Time step}
\author{Malte Stær Nissen\\
        University of Copenhagen\\
        \texttt{malte.nissen@gmail.com} \and
        Supervisor: Sune Darkner\\University of Copenhagen\\\texttt{darkner@diku.dk}}

\begin{document}
\maketitle

\tableofcontents

\clearpage

\section{Introduction}
simulations of large (order of millions of vertices) hyper elastic materials
are very time consuming. when working with materials of varying density
of vertices, the highest densities often make the most strict bounds on
size of the time step of the simulation in order to keep the materials (and
grids) stable and the discretization correct. in a ``standard'' simulation
with a global time step size we need to compute possibly unnecessarily many
computations on the coarser parts of the material caused by this bound.
when furthermore having materials with a large amount of somewhat equally
distributed vertices and smaller patches of detailed areas with higher density
of vertices, it would be an advantage to be able to perform large time steps
for the majority of the material and smaller steps for the local patches when
vertices cause instability. we call this concept local adaptive optimization
of time step, which we will study further in this report.

\section{Previous work}
As according to \cite{Gander:2013} local time stepping was first studied
in the community of ordinary differential equations (ODE) with Rice
\cite{rice:1960} developing the split Runge-Kutta methods (multi rate
Runge-Kutta methods). The concept behind these methods is the splitting of
the ODEs into multiple (two components in \cite{rice:1960}) components of
which each component needs to be integrated in different scales. A strategy
is chosen as proposed in \cite{Kvaernoe:1999} of either computing the coarser
part first (``Fastest first strategy'') followed by the finer part or vice
versa (``Slowest first strategy''). In order to perform these sequential
computations either interpolation or extrapolation of the first computation
is performed to compute the second part in each time step and the different
time step sizes can be adapted to fit the model in each step as well. See
\cite{Kvaernoe:1999} and \cite{Gear:1984} (similar strategy for linear multi
step methods) for more detailed descriptions of the two strategies.

In the partial differential equations (PDE) community, the adaptive time
stepping area was explored and developed with the goal of simulating
very specific known problems such as the (hyperbolic) wave equations
and (parabolic) heat equations instead of more general applications. We
will adhere from describing the methods further in this report, but a
short description and comparison of the specific methods can be found in
\cite{Gander:2013}. Generally the methods developed at first all make use
of the same basic concepts for making the local adaptive time stepping:
Interpolation, extrapolation, prediction and correction, which we are going to
use in the work of this report as well.

Recently newer and faster methods with local time stepping have been developed
on the basis of nonlinear PDEs and the work performed in this field in
the 80s and 90s. First the multi resolution (MR) schemes for creating
space-adaptive discretizations and refinements of these were developed, see
\cite{Berger:1984}. Then multiple MR based schemes were developed. Domingues
et al. \cite{Domingues:2008} is one of the more recent of these MR schemes.
They describe a local scale-dependent time stepping for a space-adaptive multi
resolution scheme using the finite volume method in order to obtain speed-up
using larger time-steps without violating a defined stability constraints,
which is essentially the same motivation as this report. The method is based
on an explicit Runge-Kutta method of second order. As expected the time step
size is imposed by a stability condition of the explicit Runge-Kutta on the
finest scale, which increases with the scale of the mesh and hence we are able
to increase the time step as well without violating the stability condition.

\section{Method}
We will in this report consider the 1 dimensional problem of simulating a mass
spring system, which is usually used for simulating cloth and hair etc. A mass
spring system is a system consisting of a set of vertices $V$ each having a
specified mass $m$, force $F$ and velocity $u$ and a set of springs $S$ with two
endpoints $p_f \in V$ and $p_t \in V \setminus p_f$. Each spring has a damping
factor $d$, stiffness constant $K$ and rest length $l_0$ describing the state of
the spring.

Things to write about (notes):

Damping force $F_d$:
\begin{align*}
    \Delta l &= l_{left} - l_{right} \\
    F_d      &= - C_d v(i) =  - C_d \frac{\left(v_{left} - v_{right}\right)
\Delta l}{\abs{\Delta l}}
\end{align*}

Spring force $F_s$:
\begin{align*}
    F_s = kx = -k \left( l - l_0 \right)
\end{align*}

Explicit Euler method:
\begin{align*}
    x(t+h) &= x(t) + h \frac{\partial x(t)}{\partial t}
\end{align*}

Midpoint method:
\begin{align*}
    x(t+h) = x(t) + h \left( f \left(x + \frac{h}{2} f \left(t \right) \right) \right)
\end{align*}

\section{Experiments and results}

\subsection{Switch}

\subsection{Spatial LAOT}

\subsection{Inverse}

\section{Discussion}

Adaptive timestepping based on highest acceleration/smallest grid

\section{Conclusion}


\cite{Keshav:2007}
\bibliography{bibliography}
\bibliographystyle{plain}

\end{document}
