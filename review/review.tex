\documentclass[11pt,a4paper]{article}

\usepackage[utf8]{inputenc}
\usepackage[english]{babel}
\usepackage[T1]{fontenc}
\usepackage{lmodern}

\usepackage{amsmath,amssymb,amsfonts}

\title{Review of Human Pose Tracking}
\author{Malte Stær Nissen}

\begin{document}
\maketitle

\noindent Corrections/Questions:
\begin{itemize}
    \item 1.L2: Comma before ``and video games'' (Oxford comma)
    \item 1.L13 and 2.4.1.L2: Gradient ascent/descent - which is it? Looks as if you refer to
        the same paper and the same part of the algorithm.
    \item Figure 1: Is the figure your own product or from somewhere else?
    \item Figure 1: Write which of the figures is which (good = left, bad =
        right) - it might be obvious but it's always good to be on the safe
        side.
    \item 1.L1, 2.1.L3, 2.1.L11: Shouldn't you always start with a capital letter after colon?
    \item 2.1: Why make a subsection called Introduction. Why not just write the
        entire part directly after section 2 starts?
    \item 2.3.2.L2: ``requires a shape model, described above'' (where exactly?)
    \item 2.3.3: Any reference? When you have chopped the section into
        subsections it might be a good idea to have references for all
        methods/parts
    \item 2.6.L3: ``the then state-of-the-art method''. The ``then''-part Doesn't sound correct
        in my head/mind. I have however no alternative solution at hand.
\end{itemize}

\noindent General:
\begin{itemize}
    \item Introduction: Good and sufficient describtion to get a grasp of the
        problem and the work being performed (exclusing results).
    \item When reading the litterature study (all the subsections following the
        introduction) it first seems quite messy with a
    lot of terminology but no red line. After a few times of reading the
    subsections and introduction (2.1)
    it gets more obvious what your intention/red line of the litterature study
    is. Maybe emphasize in 2.1.L7 that the overview is the other subsections of
    section 2. Would've helped me a bit when first reading the report.
    \item Litterature study level of detail good and sufficient.
    \item Consistency in writing style (sections/subsections): Some places
    (2.4) you describe (name) the following subsections in the overall
    section. Other places (2.3.2, 2.3, 2.5) you only write that there are X
    (number) of methods/subsections, not what they are.
    \item Maybe some sort of conclusion of the litterature study could be added
        at the end? Some comments on what exactly you're going to implement/use of all
        the stuff you present. I do know that you've described what you're going
        to implement earlier but a nice round-up of exact references might be an
        idea?
\end{itemize}


\end{document}

